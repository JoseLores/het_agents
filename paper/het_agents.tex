\documentclass[a4paper,12pt]{article}
\usepackage[margin=1in]{geometry}
\usepackage[utf8]{inputenc}
\usepackage{multirow}
\usepackage{booktabs}
\usepackage{amsmath}
\usepackage{verbatim}
\usepackage{setspace}
\spacing{1.5}
\usepackage{graphicx}
\usepackage{tabularx}
\usepackage{dcolumn}
\usepackage{indentfirst}
%\usepackage{subfigure}
\usepackage[titletoc,title]{appendix}
\usepackage{apacite}
\usepackage{natbib}
\setcitestyle{aysep={}} %remove comma between author and year when using citep{}.
\usepackage{color}
\usepackage{longtable}
\usepackage{lscape}
\usepackage{pdflscape}
\usepackage{caption}
\usepackage{subcaption}
\graphicspath{{../../Output/Figures/}}
\newcommand{\annote}[1]{\parbox{\textwidth}{\renewcommand{\baselinestretch}{1.0}\vspace{12pt} \footnotesize Notes: #1}}
\renewcommand{\vec}[1]{\mathbf{#1}}
\usepackage[usenames,dvipsnames,svgnames,table]{xcolor}
\usepackage{libertine}
\usepackage{hyperref}
\hypersetup{
     colorlinks   = true,
     urlcolor	  = black,
     citecolor    = purple,
     linkcolor 	  = black
}
\begin{document}

\title{Solve Steady State for Heterogeneous Agents Models \thanks{Jose Lores Diz, University of Bonn. Email: \href{mailto:ldpepe99@gmail.com}{\nolinkurl{ldpepe99 [at] gmail [dot] com}}.}}

\author{Jose Lores Diz}

\date{
    {\bf Preliminary -- please do not quote}
    \\[1ex]
    \today
}

\maketitle


\begin{abstract}
I employ various numerical methods seen in classes at University of Bonn, such as the endogenous grid method \cite{carroll2006} and Young's \cite{young2015} method, to solve for the steady state of two heterogeneous agents models. The models are both inspired in \citet{aiyagari1994} but differ on their income processes, market structures and saving constrains (in one agents can get indebted up to some point).

I systematically analyze the behavior of the supply and demand curves for funds and the impact of interest rates on the wealth distribution. To this end, I develop a suite of visualization tools that display the supply and demand curves, Lorenz curves, and wealth distributions.

In addition to comparing the performance of different numerical methods, we also explore the sensitivity of the model to parameter changes. Throughout the project, I emphasize the importance of robustness and reproducibility, incorporating test-driven development and version control for efficient and accurate implementation. The project set up follows \cite{GaudeckerEconProjectTemplates} and makes a heavy use of \cite{numpy} and \cite{estimagic}.
\end{abstract}

\clearpage


\section{Benchmarking algorithms with estimagic.} % (fold)
\label{1}

\begin{figure}[htbp]
    \centering
        \includegraphics[width=0.7\textwidth]{../bld/python/figures/run_time_plot.png}
        \caption{Average run time of each solver.}
        \label{fig:1}
\end{figure}

\vspace{1cm}

\begin{figure}[htbp]
        \centering
        \includegraphics[width=0.7\textwidth]{../bld/python/figures/iterations_plot.png}
        \caption{Equilibrium of the capital markets in the standard model (no incomplete markets).}
        \label{fig:2}
\end{figure}


\clearpage

\section{Models equilibrium.} % (fold)

\begin{figure}[htbp]
        \centering
        \includegraphics[width=0.7\textwidth]{../bld/python/figures/incomplete_supply_demand_plot.png}
        \label{fig:3}
        \caption{Incomplete markets, solver finds the equilibrium in a bound.}
\end{figure}

\vspace{1cm}

\begin{figure}[htbp]
        \centering
        \includegraphics[width=0.7\textwidth]{../bld/python/figures/standard_supply_demand_plot.png}
        \caption{Number of iterations that each solver needs.}
        \label{fig:4}
\end{figure}

\clearpage

\section{Standard model distributions with five income states.}

\begin{figure}[h]

    \centering
    \includegraphics[width=0.7\textwidth]{../bld/python/figures/wealth_distrib_plot.png}

    \caption{\emph{Wealth distribution in the steady state:}  from the model with standard markets and allowing agents to get indebted. }
    \label{fig:5}

\end{figure}

\vspace{1cm}

\begin{figure}[h]

    \centering
    \includegraphics[width=0.7\textwidth]{../bld/python/figures/lorenz_plot.png}

    \caption{\emph{Lorenz Curve} }
    \label{fig:6}

\end{figure}


\clearpage

\bibliographystyle{apacite}
\begin{singlespace}
\bibliography{refs}
\end{singlespace}



\end{document}
